\begin{DoxyAuthor}{Author}
Peter Neuss
\end{DoxyAuthor}
\hypertarget{index_intro}{}\section{Introduction}\label{index_intro}
The primary purpose of this library is to provide peer-\/to-\/peer connections for MyMed applications.

There are two main software packages, and two levels of wrapper.

The low-\/level software packages is written in C, and consists of two files :
\begin{DoxyItemize}
\item p2pConnApi.c -\/ this uses the open-\/source pjnath library to create UDP-\/style peer-\/to-\/peer connections
\item rsComm.c -\/ this communicates with a MyMed Rendezvous Server (RS) to bootstrap the peer-\/to-\/peer connections.
\end{DoxyItemize}

The \hyperlink{namespacePseudoTcp}{PseudoTcp} software package is written in C++. It assumes a UDP-\/style connection adhering to the IConnection interface. Using this it implements a reliable message-\/based connection. In particular, it provides :
\begin{DoxyEnumerate}
\item Retransmission of lost messages.
\item Elimination of duplicates.
\item Ordering of messages.
\end{DoxyEnumerate}

The first wrapper integrates the \hyperlink{namespacePseudoTcp}{PseudoTcp} package with the low-\/level package. It consists of four files :
\begin{DoxyItemize}
\item IceConnection.cpp/.hpp -\/ This is an object which implements IConnection using the low-\/level p2p connection
\item IceWrapper.cpp/hpp -\/ This providec a C++ facade for the low-\/level functionality and the \hyperlink{namespacePseudoTcp}{PseudoTcp}. A MyMed application written in C++ would use this as an API for the library.
\end{DoxyItemize}

The second wrapper provides a Java interface for the functionality and is implemented using JNI. It consists of the following files :
\begin{DoxyItemize}
\item JavaIceWrapper.java -\/ This loads the dynamic library and declares the API
\item \hyperlink{mymed__JavaIceWrapper_8h_source}{mymed\_\-JavaIceWrapper.h} -\/ This is created automatically and declares the native function headers
\item JavaIceWrapper.cpp -\/ This file contains the definitions of the native functions.
\end{DoxyItemize}\hypertarget{index_javaint}{}\section{Java Interface}\label{index_javaint}
\hypertarget{index_create}{}\subsection{How to create the dynamic library}\label{index_create}
The shell file 'makeDynLib.sh' will create the library file. It consists of five steps:
\begin{DoxyEnumerate}
\item Compile the JavaIceWrapper.java file creating a JavaIceWrapper.class file
\item Create the \hyperlink{mymed__JavaIceWrapper_8h_source}{mymed\_\-JavaIceWrapper.h} file automatically from the class file
\item Compile the low-\/level .c files into object files appropriate for a dynamic library
\item Compile the .cpp files into object files appropriate for a dynamic library
\item Combine the object files into a dynamic library named libmymed.so.1.0.1
\end{DoxyEnumerate}\hypertarget{index_connect}{}\subsection{How to connect to the library from Java}\label{index_connect}

\begin{DoxyEnumerate}
\item Include the JavaIceWrapper.java file in your Java application (it is in the mymed package)
\item Create a symbolic link to the libmymed.so.1.0.1 library called libmymed.so
\item make sure the Java compiler can find the libmed.so file (either by placing it in a standard location or through the command-\/line parameter -\/Djava.library.path)
\end{DoxyEnumerate}\hypertarget{index_api}{}\subsection{The Java API}\label{index_api}
The JavaIceWrapper class reference details the API. See also the Chat application written in Java/Swing for an example of how to use the library.\hypertarget{index_Documentation}{}\section{Documentation}\label{index_Documentation}
This documentation has been created by doxygen. To generate it, in the IceIntegration directory execute the command: doxygen DoxyFile



 